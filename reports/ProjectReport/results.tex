\section{Results}
A schedule that was generated by the system can be found in
\autoref{app:schedule}. As can be seen in this schedule, There are no
conflicts in that no room is allocated twice in the same timeslot and no
lecturer has to give two lectures at the same time. We can also see that
all the groups of students fit into the rooms\footnote{The sizes for the
    rooms can be found on
    \url{https://github.com/XeryusTC/aa/blob/master/locations.yaml} and the
    sizes of the groups of students on
    \url{https://github.com/XeryusTC/aa/blob/master/teachers.yaml}}.

However, the system failed in that it did not schedule a lecture for
Computational Cognitive Neuroscience by Marieke van Vugt. If we look at
the schedule, we can see that every time that there is a room left in a
time slot, we can see that in all of theses cases Marieke van Vugt already
gives a lecture, and not being able to be at two locations at once cannot
give a second lecture at the same time. 

\section{Conclusion}
Our system is able to construct a conflict free schedule, proving that our
approach to scheduling is a viable one. Furthermore, it is currently
possible for a human to look at the end state of the argumentation
framework and see why a certain agent got the room allocated, meaning it is
possible to now give a reasoning for why the schedule is the way it is.
This was one of the problems we wanted to solve.

The fact that the system was not able to schedule all lectures probably
follows from the fact that we do not have a lot of rooms, being able to
only schedule three parallel lectures, while in a real scenario there are a
lot more rooms available, mostly for smaller groups. If we were to include
these rooms as well, more lectures should be able to fit into the schedule
without problems.

Another explanation is that we over scheduled lecturers in that with our
current inputs, they give more courses than they normally would in a
semester. Because of this, the lecturers spend a lot of their time on
teaching courses, making it harder for them to fit new courses in. 
