\subsection{State of the Art}
We looked into the state of the art of scheduling software and found that 
Syllabus Plus, a product developed by Scientia, is used by a large number 
of colleges and universities around the world, including the University of 
Groningen \cite{SyllabusPlus}. Syllabus Plus is described as an artificial 
intelligence solution, with its implementation resembling a constraint 
satisfaction problem solver. The program operates by collection available 
data, which consists of foundation and student data. The former contains 
non-dynamic information about teacher staff and locations, while the latter 
involves dynamic data, such as student enrollments and course requirements. 
After a refinement phase for all data is performed, the program iterates 
through constraint reduction cycles, in which the hard constraints, soft 
constraints, rules and preferences are adjusted to some degree, until the 
stop criteria of the program are reached. The result is a working schedule 
that satisfies the target constraints.

The state of the art for using argumentation in the scheduling process is 
described in a paper by \cite{Kuo:jc}. Here, a course-scheduling 
negotiating system is presented in which agents, who are representing 
teachers, are involved in the scheduling process. This involvement consists 
of negotiating and arguing with other agents, in order to reach a consensus 
about a particular room allocation or to change an agent's belief. Apart 
from beliefs, agents also posses individual preferences, to which a number 
of proposals are generated for every agent during the proposal generator 
phase of the algorithm. Proposals are abstract, but can for example take 
the form of agent \textit{A} claiming room \textit{1} for course 
\textit{alpha} on time slot \textit{X}. Next, during the negotiation phase, 
every agent selects its most effective proposal out of the total list, by 
calculating an utility value for all proposals and selecting the best 
scoring option. The utility value is based on particular preferences for 
issues belonging to a given proposal. Every agent will then present their 
favorite proposal and will try to negotiate with other agents in order to 
reach a consensus about a single proposal. If such a consensus is indeed 
reached, the scheduling process is completed for the rooms, courses and 
agents involved in the proposal. If however no consensus is reached, the 
agent with the lowest so-called risk value is selected to concede. This 
risk value is evaluated for every agent by taking the utility value of the 
proposal chosen by agent minus the maximal cost of conceding ones own 
proposal in favor of that of another. If there are no more proposals to be 
selected that have not been used during the negotiation phase this round, 
the argumentation phase is started. Else, the conceding agent is prompted 
to select an available proposal and resume the negotiation phase. In the 
argumentation phase, the conceding agent chooses the agent with the lowest 
risk value (itself excluded) to pass to and fro arguments with. 

 Based on their beliefs and differences their on, 
the agents will created arguments trying to change each other\'s beliefs. 
Based on the outcome of this, particular beliefs for the respective agents 
will be updated in the belief evaluation phase. Afterwards, algorithm will 
go back into the negotiation phase. The entire process will continue until 
consensus is reached during a negotiation phase.