\subsection{State of the Art}
With regard to scheduling software, Syllabus Plus is said to be used by a large number of colleges and universities around the world, including the University of Groningen \cite{SyllabusPlus}. Syllabus plus is described as an artificial intelligence solution and can be seen as a constraint satisfaction problem solver. The program operates by first taking non-dynamic data, such as teacher stuff, room stuff, building stuff, called foundation data. On top of this dynamic data is loaded such as student enrollments and course requirements. Data is refined and loaded into problem solver. This solver iterates over the different soft constraints, hard constraint and preferences and adjusts stuff until a set amount of constraints are satisfied and a working schedule is outputted.  

The state of the art for using argumentation in the scheduling process is described in a paper by \cite{Kuo:jc}. Here an algorithm is proposed that takes the teacher as agents and lets them argue over stuff. Every agent is created with beliefs and preferences. Based on this a set number of proposals are generated per agent, with proposals being certain elements of the schedule. Every agent calculates an utility value based on its preferences for every proposal (so also proposals created by other agents), and choses the proposal with the highest utility. The negotiation phase is now started, in which every agent will present its selected proposal. If there is a concensus about a particular proposal, the algorithm is finished for this round. If there is no concensus, the agent with the lowest risk value, which is based on how much utility is lost conceding, will concede and the argumentation phase is started. Here, the conceding agent will argue with the agent with the lowest risk value  that it hasent argued before in this round yet. Based on their beliefs and differences their on, the agents will created arguments trying to change each other\'s beliefs. Based on the outcome of this, particular beliefs for the respective agents will be updated in the belief evaluation phase. Afterwards, algorithm will go back into the negotiation phase. The entire process will continue untill consensus is reached during a negotiation phase.