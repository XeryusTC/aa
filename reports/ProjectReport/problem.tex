\section{Introduction}
It is 8:43 am. The combination of early-morning coffee and a constant
breeze of cold air keeps you away from the sweet slumber you had an hour
ago. While you zigzag effortlessly between the seemingly infinite stream
of fellow bicycle-riders, you ponder why in the first place you would have
a lecture planned on Friday, nine o'clock. You manage to barely make it on
time. Entering the room, you can see that the majority of people have
already arrived, more or less awake. After turning on the beamer and
connecting your laptop, you start facing the crowd. ``Good morning
everyone!''

Timetable schedules can be met with occasional criticisms from its intended
audience. This form of feedback is often based on individual preferences
and needs, and is usually intended to adjust a schedule to better fit the
individual instead of the group. While the concept of arguing about one's
own preferences in this scenario might seem selfish, it does provide a way
to augment the process of scheduling.

Room scheduling can be seen as a variant of resource allocation, as it
usually pertains to having a selection of time-slots, rooms and courses.
These last two each contain limitations and requirements respectively, such
as the number of students that can be seated or enrolled or whether a
beamer is, or should be present. The problem of room scheduling is to find
configurations of rooms for lectures, practicals or meetings in a specific
time-frame, where an utility-value based on these limitations and
requirements is maximum for all affected agents. This list of agents
includes teachers, students and other individuals that are involved in the
schedule.            

The use of agent argumentation could be beneficial to solving the schedule
problem, as it helps to put the utility-value of agents into more concrete
terms by placing the focus on the individual agent. Agents would be able to
argue about each other's preferences and constraints, which incidentally
should also lead to a reduction of the amount of criticisms on schedules,
because one's preferences would now be included in the overall process. 

In this report a scheduling solution is presented that uses agent
argumentation as basis. Teachers are seen as agents, who have the ability
to make claims on rooms and can create attack, support and undercut
arguments for or against other claims and arguments. The arguments can be
based on the preference-weights of agents, as well as the attributes of
agents, courses and rooms. Based on the results of these arguments per room
a schedule is constructed that assigns rooms to specific courses for a
given time-frame.
