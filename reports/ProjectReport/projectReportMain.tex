\documentclass[a4paper]{article}
\usepackage{bnaic}
\usepackage{latexsym}
\usepackage{color}
\usepackage{listings}
\usepackage{amsmath}
\usepackage{amsthm}
\usepackage[pdftex]{graphicx}
\usepackage{subfigure}
\usepackage[numbers]{natbib}
\usepackage{hyperref}
\usepackage[all]{hypcap}

\definecolor{White}{rgb}{1,1,1}
\definecolor{Blue}{rgb}{0,0,0.5}
\definecolor{Green}{rgb}{0,0.75,0.0}
\definecolor{LightGray}{rgb}{0.6,0.6,0.6}
\definecolor{DarkGray}{rgb}{0.3,0.3,0.3}
\lstset{language=Python,
   basicstyle=\ttfamily\small,
   breaklines=true,
   keywordstyle=\bfseries\color{Blue},
   commentstyle=\itshape\color{LightGray},
   stringstyle=\color{Green},
   numbers=left,
   numberstyle=\tiny\color{DarkGray},
   stepnumber=1,
   numbersep=10pt,
   backgroundcolor=\color{White},
   tabsize=2,
   showspaces=false,
   showstringspaces=false,
   captionpos=b}

%Boldface text for type writer font
\usepackage{bold-extra} %\DeclareFontShape{OT1}{cmtt}{bx}{n}{<5><6><7><8><9><10><10.95><12><14.4><17.28><20.74><24.88>cmttb10}{}

%Break words properly at the end of a line (which isn't sloppy...)
\sloppy

%Use command \exercise for each exercise
\newcounter{exerciseCount}
\setcounter{exerciseCount}{0}
\newcommand{\exercise}[1]{\addtocounter{exerciseCount}{1} \noindent \medskip {\large \textsf{\textbf{Exercise \arabic{exerciseCount} \--- #1}}} \par}
\renewcommand{\theenumi}{\textsf{\textbf{\alph{enumi}}}}

%Use command \code for code snippets
\newcommand{\code}[1]{\textnormal{\texttt{#1}}}

\theoremstyle{definition}
\newtheorem{defn}{Definition}

\title{\textbf{\huge Arguing Agents\\An argumentation based solution to
        scheduling}}
\author{Pim van der Meulen\and Ren\'e Mellema \and Xeryus Stokkel}
\date{\textit{Institute of Artificial Intelligence, University of
        Groningen}}

\begin{document}
\ttl{}
\thispagestyle{empty}

\begin{abstract}
    \noindent
\end{abstract}

%Section 1: Introduction
\input{introduction.tex}

%Section 6: Problem
\section{Introduction}
It is 8:43 am. The combination of early-morning coffee and a constant breeze of cold air keeps you away from the sweet slumber you had an hour ago. While you zig-zag effortlessly between the seemingly infinite stream of fellow bicycle-riders, you ponder why in the first place you would have a lecture planned on friday, nine o'clock. You manage to barely make it on time. Entering the room, you can see that the majority of people have already arrived, more or less awake. After turning on the beamer and connecting your laptop, you start facing the crowd. "Good morning everyone!"    

Timetable schedules can be met with occasional criticisms from its intended audience. This form of feedback is often based on individual preferences and needs, and is usually intended to adjust a schedule to better fit the individual instead of the group. While the concept of arguing about one's own preferences in this scenario might seem selfish, it does provide a way to augment the process of scheduling.

Room scheduling can be seen as a variant of resource allocation, as it usually pertains to having a selection of time-slots, rooms and courses. These last two each contain limitations and requirements respectively, such as the number of students that can be seated or enrolled or whether a beamer is, or should be present. The problem of room scheduling is to find configurations of rooms for lectures, practicals or meetings in a specific time-frame, where an utility-value based on these limitations and requirements is maximum for all affected agents. This list of agents includes teachers, students and other individuals that are involved in the schedule.            

The use of agent argumentation could be beneficial to solving the schedule problem, as it helps to put the utility-value of agents into more concrete terms by placing the focus on the individual agent. Agents would be able to argue about each other's preferences and constraints, which incidently should also lead to a reduction of the amount of critizisms on schedules, because one's preferences would now be included in the overall process. 

In this report a scheduling solution is presented that uses agent argumentation as basis. Teachers are seen as agents, who have the ability to make claims on rooms and can create attack, support and undercut arguments for or against other claims and arguments. The arguments can be based on the preference-weights of agents, as well as the attributes of agents, courses and rooms. Based on the results of these arguments per room a schedule is constructed that assigns rooms to specific courses for a given time-frame.


%Section 2: State of the Art
\subsection{State of the Art}
With regard to scheduling software, Syllabus Plus is said to be used by a large number of colleges and universities around the world, including the University of Groningen \cite{SyllabusPlus}. Syllabus plus is described as an artificial intelligence solution and can be seen as a constraint satisfaction problem solver. The program operates by first taking non-dynamic data, such as teacher stuff, room stuff, building stuff, called foundation data. On top of this dynamic data is loaded such as student enrollments and course requirements. Data is refined and loaded into problem solver. This solver iterates over the different soft constraints, hard constraint and preferences and adjusts stuff until a set amount of constraints are satisfied and a working schedule is outputted.  

The state of the art for using argumentation in the scheduling process is described in a paper by \cite{Kuo:jc}. Here an algorithm is proposed that takes the teacher as agents and lets them argue over stuff. Every agent is created with beliefs and preferences. Based on this a set number of proposals are generated per agent, with proposals being certain elements of the schedule. Every agent calculates an utility value based on its preferences for every proposal (so also proposals created by other agents), and choses the proposal with the highest utility. The negotiation phase is now started, in which every agent will present its selected proposal. If there is a concensus about a particular proposal, the algorithm is finished for this round. If there is no concensus, the agent with the lowest risk value, which is based on how much utility is lost conceding, will concede and the argumentation phase is started. Here, the conceding agent will argue with the agent with the lowest risk value  that it hasent argued before in this round yet. Based on their beliefs and differences their on, the agents will created arguments trying to change each other\'s beliefs. Based on the outcome of this, particular beliefs for the respective agents will be updated in the belief evaluation phase. Afterwards, algorithm will go back into the negotiation phase. The entire process will continue untill consensus is reached during a negotiation phase.

%Section 3: New Idea
\section{Method}

\subsection{Argumentation Framework}
Our argumentation framework is based on Dung's argumentation semantics
\cite{dungargfund} with special focus on the grounded semantics. In order
to make the arguments seem more natural, it is extended to also work for
support and undercuts. This method was chosen, because it was easy to
compute, but gave a powerful and easy to use interface.

Like in~\cite{dungargfund}, the basis for the framework is a graph, where
attack and support relations are given by the edges and the arguments
themselves are the nodes. The undercutting attacks go from arguments to
edges.\footnote{Because the graph library in Python that we used did not
    allow for edges that went to edges, a structural node was inserted in
    each attack and support relation} 
In the current framework it is only
possible to undercut attack and support relations, not undercutting
relations themselves. This is because of limitations of the current
implementation, not of the theoretical framework. Each relation also has a
weight, or a strength, which for supporting relationships is positive and
for attacking relationships is negative.

Because we also included support and undercutting, the grounded semantics
were extended to account for this.
\begin{defn}
    An argument is grounded iff
    \begin{enumerate}
        \item It is not attacked by a grounded argument that is not
            undercut, or
        \item The sum of the weights of the grounded, not undercut attack
            and support relations is larger than 0
    \end{enumerate}
\end{defn}

In simpler terms, this means that an argument is grounded if it has no
attacks, of if it has more support than attacks. The first clause is needed
to make sure that arguments that are not attacked are grounded. If these
arguments would not be grounded, then they do not count to determining if
the arguments they have outgoing relations to are grounded. This means that these
arguments do not have incoming attack or support relations. Therefore they
do not have more support than attacks and are not grounded either, meaning
that no argument is grounded. 

We now also need a definition for undercut, since this is important for our
definition of grounded.
\begin{defn}
    An attack or support relation is undercut iff the sum of the weights of the
    undercutting relations coming from grounded arguments is equal to or
    lower than 0.
\end{defn}


\subsection{Scheduling}
To create a schedule we need several pieces of information. We need to know
which rooms are available and how many students they can seat. Most courses
require a beamer while some others do not, so we also keep track of whether
there is a beamer in the room. Next we need to know the various timeslots that
each course will occupy. So a several start times and end times are given to
create blocks in which the courses will take place, these time slots are always
as shown in \autoref{tbl:timeslots}.

The final information we require is what courses the lecturers teach. For each
course we keep track of who teaches the course, how many students there are who
follow the courses, whether the course requires the use of a beamer, and how
many lectures need to be scheduled for that course. In addition to this we also
want to know the room preferences of the teachers. For this we use a number
between 0 and 1, where 0 represents that the teacher never wants to teach in
that room while 1 represents that the teacher loves the room and always wants to
teach there if he has the opportunity.

To successfully create a schedule these elements need to be combined in a way
that does not create any conflicts. In this case conflicts would be when a
teacher has to teach at two different places at the same time, multiple lectures
take place in the same room at the same time, there are more students than there
are seats in a room, or a room is unfit for the course because there is no
beamer in it while it is required.

To create the conflict free schedule we first create a $|R| \times |T| \times
5$ matrix, where $R$ is the set of rooms, $T$ is the set of available time
slots (see \autoref{tbl:timeslots}). The constant 5 comes from the number of
days we want to schedule for, in this case the days Monday through Friday. Next
we pick a random time, room and day combination from the matrix and ask each
lecturer agent if it is interested in teaching a course at this moment. The
agent decides if this is the case based on its availability, so if it is
occupied with teaching another course at that point in time then it is clearly
not able to teach at that moment as well. The agent will also check whether it
teaches a course that will fit in the room, if all courses that an agent teaches
have more students enrolled than there are seats in the room then the agent is
also not interested. After asking all agents an argument framework is created
for this room, all interested agents make a starting argument which we call a
claim. These claims start by attacking each other. Each agent determines for
itself what course it wants to start arguing for, they will always pick the
course with the most students that fits in the room that still has unscheduled
lectures.

The argument takes place over the course of several iterations. Each iteration
we ask each agent if they want to make an argument based on the current state of
the argumentation framework. Agents will support their own claim by making a
support argument. These support arguments can be one of a few types. The first
type is the size argument, in this case an agent makes the argument that the
class that they want to schedule has a certain number of students. The second
type is a beamer argument, where agents will say that they require a beamer or
not. The third type is an agent's preference for the room. After making initial
arguments agents are also allowed to start attacking and undercutting each
other's arguments. Agents will always undercut each others arguments if they
make the same argument, i.e. when the number of students in their course is the
same or they both require a beamer. When the agent has a better argument, like
their class being larger, than another agent it will also undercut the argument.

This process continues until no more arguments have been made. At this point we
calculate the grounded set of the argumentation framework. Often this will
result in just one claim being in the grounded set. The agent who made this
claim gets to schedule the course that they made the claim for. So this
particular room/time/day combination is now scheduled and won't be discussed
again. The winning agent keeps track of which courses have had lectures
scheduled so that they will not schedules more lectures than necessary for each
course. If there are multiple claims in the grounded set then the winner is
picked at random between these claims. When this happens it means that all
agents made equal arguments so there is no clear winner.

\begin{table}
	\centering
	\caption{Overview of the time slots that courses can be planned in.}
	\label{tbl:timeslots}
	\begin{tabular}{l|l|l}
		& Start time & End time \\ \hline
		Block 1 & 09:00 & 11:00 \\
		Block 2 & 11:00 & 13:00 \\
		Block 3 & 13:00 & 15:00 \\
		Block 4 & 15:00 & 17:00
	\end{tabular}
\end{table}



%Section 4: Results
\section{Results}
A schedule that was generated by the system can be found in
\autoref{app:schedule}. As can be seen in this schedule, There are no
conflicts in that no room is allocated twice in the same timeslot and no
lecturer has to give two lectures at the same time. We can also see that
all the groups of students fit into the rooms\footnote{The sizes for the
    rooms can be found on
    \url{https://github.com/XeryusTC/aa/blob/master/locations.yaml} and the
    sizes of the groups of students on
    \url{https://github.com/XeryusTC/aa/blob/master/teachers.yaml}}.

However, the system failed in that it did not schedule a lecture for
Computational Cognitive Neuroscience by Marieke van Vugt. If we look at
the schedule, we can see that every time that there is a room left in a
time slot, we can see that in all of theses cases Marieke van Vugt already
gives a lecture, and not being able to be at two locations at once cannot
give a second lecture at the same time. 


%Section 5: Relevance
\section{Discussion}
Our system is able to construct a conflict free schedule, proving that our
approach to scheduling is a viable one. Furthermore, it is currently
possible for a human to look at the end state of the argumentation
framework and see why a certain agent got the room allocated, meaning it is
possible to now give a reasoning for why the schedule is the way it is.
This was one of the problems we wanted to solve.

The fact that the system was not able to schedule all lectures probably
follows from the fact that we do not have a lot of rooms, being able to
only schedule three parallel lectures, while in a real scenario there are a
lot more rooms available, mostly for smaller groups. If we were to include
these rooms as well, more lectures should be able to fit into the schedule
without problems.

Another explanation is that we over-scheduled lecturers in that with our
current inputs, they give more courses than they normally would in a
semester. Because of this, the lecturers spend a lot of their time on
teaching courses, making it harder for them to fit new courses in. 

To create the schedule we randomly pick rooms from the timetable and let agents
argue about who gets to lecture at that time slot. The random nature has several
implications for the way that the schedule is constructed. The lecture that wins
is not always a lecture that fits well into that room, it is possible that a
course with 20 students is placed in a 120 seat room because the lecturer was
the only agent that was interested in that time slot. This means that the
lecturer will teach a course to a mainly empty room, something which can be
quite demotivating even if every student shows up. Ideally a course is taught in
a room that fits best, that is, there should be as few empty seats as possible.
To enable this the slots should be filled in a more systematic way, the best
solution would be to start filling the smallest rooms with courses first. This
way small classes will always be in the smallest possible room where they fit
since the agents will try to claim whatever room that a course fits in.

Another characteristic of the random scheduling of rooms is that the system is
less likely to create the best schedule, or that it will even find a complete
schedule at all, even if it exists. A complete schedule in this case means that
all lectures that needed to be scheduled are scheduled without any conflicts.
The random nature allows for an ordering that will lead to scheduling conflicts
for single agents. If there are two agents, of which agent 1 teaches a 60 people
class, and agent 2 teaches a 120 people class, and there are two rooms
available: room 1 seats 80 people while room 2 seats 130 people. Say that we
schedule for room 2 first and agent 1 is the only grounded claim because that
agent has a higher preference for that room. It is then impossible to schedule
the class of agent 2 since the only room available seats only 80 people. So it
would've been better to schedule agent 1's class in room 1. Unfortunately
because of the random nature of our scheduling we can have this happen. A more
systematic approach could solve these kinds of conflicts before they even occur.
Scheduling the smallest rooms first would solve this. Allowing agents to make
an argument that the currently discussed room is the minimum size for their
course will also prevent this from happening, even without needing to change the
order in which the timetable is filled. The downside of this is that agents need
to know what size rooms are available, this is information that they currently
do not possess.

A different problem with our current system is that it can schedule lectures
for the same group of students but for different courses and with
different lecturers and the same time. This is because it currently does
not take which students attend which lectures into account. We propose two
methods of solving this. The first would be to make sure that lecturers
cannot make a claim for a course in a time slot if in the same time slot a
course which the same students need to attend is already allocated. Since
the rooms and scheduled after one another, this would remove that problem,
and this would work particularly well for programmes with a lot of
compulsory courses. 

A problem with this approach is that in real life lectures tend to overlap,
even for the same groups of students if there are a lot of optional courses
which cater to different groups of students. An approach that would work
well to solve this problem is to give claims for courses of which the
students already have another course a disadvantage by adding an attack
relation to these claims. Then they would need more support than otherwise
necessary in order to win that round. Seeing as both of these two
approaches have their own drawbacks in different scenarios, a combination
of the two might also solve the problem.  In this case the first approach
would be used if a compulsory course is scheduled or made a claim for and
the second case is used if the courses are optional. This would mimic the
current behaviour of allowing overlap if the courses are optional.

The agent based approach we proposed allows for a divide-and-conquer approach to
scheduling problems. CSP solvers have a global overview of the problem and can
use that to find an optimal complete solution. However it also takes a long time
while an optimal solution is not always necessary, it is more important that
there is a working schedule than there is to make the best schedule possible.
Because we divide the problem into a sub-discussion for each time slot/room
combination we are able to speed up the process, finding a complete schedule in
just a few seconds. To do this we sacrifice optimality as discussed above, we
don't always schedule the best lecture in the best fitting room. Needing only
several seconds instead of hours gives us a great advantage over CSP solvers.
Unfortunately our implementation is less complex and can not deal with all
possible constraints that commercial packages have, if these constraints were to
be added it may well be that our time to find a solution will likely increase
exponentially.

\subsection{Relevance}
Our method of reserving rooms at certain times can be generalized to a method to
schedule the use of resources. Rooms can be swapped out for any other resource.
If one wants to schedule the usage of a telescope, like the Hubble Space
Telescope for example then there are also many constraints and preferences that
different scientific teams have. Finding a good schedule that satisfies these
constraints is a hard problem that has historically been tackled by CSP solvers
\cite{johnston1994spike}, just like the course scheduling problem. An agent
based approach like ours could also be used to tackle this scheduling task.

\subsection{Future research}
An interesting future direction of research is checking if people prefer
the schedules generated using our new system instead of the old constraint
satisfaction problem way. This could be done by feeding both systems
the same information and then asking lecturers to specify which schedule
they prefer. In order to make it even more interesting, it would also be
possible to provide an explanation for why certain lectures are scheduled
the way they are, and see if this improves peoples appreciation of the
schedules.

Another interesting area of research would be to see if the new method
could be used for general resource allocation instead of just for
scheduling. This could help in multi agent systems where multiple agents
want to use the same resource, for example a door or a hallway, but where
it is hard to find a reasonable way to give an utility for the use of the
resource. In this case, the agents could argue for why they need to use the
resource, without having to know why the other agent needs to use it, since
they will learn this during the process of argumentation. It also means
that the argumentation can happen locally, since only the agents that want
to use the resource at that time need to be involved.


\bibliographystyle{plainnat}
\bibliography{literature}
\end{document}
