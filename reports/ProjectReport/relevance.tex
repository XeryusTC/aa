\section{Discussion}
Our system is able to construct a conflict free schedule, proving that our
approach to scheduling is a viable one. Furthermore, it is currently
possible for a human to look at the end state of the argumentation
framework and see why a certain agent got the room allocated, meaning it is
possible to now give a reasoning for why the schedule is the way it is.
This was one of the problems we wanted to solve.

The fact that the system was not able to schedule all lectures probably
follows from the fact that we do not have a lot of rooms, being able to
only schedule three parallel lectures, while in a real scenario there are a
lot more rooms available, mostly for smaller groups. If we were to include
these rooms as well, more lectures should be able to fit into the schedule
without problems.

Another explanation is that we over scheduled lecturers in that with our
current inputs, they give more courses than they normally would in a
semester. Because of this, the lecturers spend a lot of their time on
teaching courses, making it harder for them to fit new courses in. 

To create the schedule we randomly pick rooms from the timetable and let agents
argue about who gets to lecture at that time slot. The random nature has several
implications for the way that the schedule is constructed. The lecture that wins
is not always a lecture that fits well into that room, it is possible that a
course with 20 students is placed in a 120 seat room because the lecturer was
the only agent that was interested in that time slot. This means that the
lecturer will teach a course to a mainly empty room, something which can be
quite demotivating even if every student shows up. Ideally a course is taught in
a room that fits best, that is, there should be as few empty seats as possible.
To enable this the slots should be filled in a more systematic way, the best
solution would be to start filling the smallest rooms with courses first. This
way small classes will always be in the smallest possible room where they fit
since the agents will try to claim whatever room that a course fits in.

Another characteristic of the random scheduling of rooms is that the system is
less likely to create the best schedule, or that it will even find a complete
schedule at all, even if it exists. A complete schedule in this case means that
all lectures that needed to be scheduled are scheduled without any conflicts.
The random nature allows for an ordering that will lead to scheduling conflicts
for single agents. If there are two agents, of which agent 1 teaches a 60 people
class, and agent 2 teaches a 120 people class, and there are two rooms
available: room 1 seats 80 people while room 2 seats 130 people. Say that we
schedule for room 2 first and agent 1 is the only grounded claim because that
agent has a higher preference for that room. It is then impossible to schedule
the class of agent 2 since the only room available seats only 80 people. So it
would've been better to schedule agent 1's class in room 1. Unfortunately
because of the random nature of our scheduling we can have this happen. A more
systematic approach could solve these kinds of conflicts before they even occur.
Scheduling the smallest rooms first would solve this. Allowing agents to make
an argument that the currently discussed room is the minimum size for their
course will also prevent this from happening, even without needing to change the
order in which the timetable is filled. The downside of this is that agents need
to know what size rooms are available, this is information that they currently
do not possess.

Another problem with our current system is that it can schedule lectures
for the same group of students but for different courses and with
different lecturers and the same time. This is because it currently does
not take which students attend which lectures into account. We propose two
methods of solving this. The first would be to make sure that lecturers
cannot make a claim for a course in a time slot if in the same time slot a
course which the same students need to attend is already allocated. Since
the rooms and scheduled after one another, this would remove that problem,
and this would work particularly well for programmes with a lot of
compulsory courses. 

A problem with this approach is that in real life lectures tend to overlap,
even for the same groups of students if there are a lot of optional courses
which cater to different groups of students. An approach that would work
well to solve this problem is to give claims for courses of which the
students already have another course a disadvantage by adding an attack
relation to these claims. Then they would need more support than otherwise
necessary in order to win that round. Seeing as both of these two
approaches have their own drawbacks in different scenarios, a combination
of the two might also solve the problem.  In this case the first approach
would be used if a compulsory course is scheduled or made a claim for and
the second case is used if the courses are optional. This would mimic the
current behaviour of allowing overlap if the courses are optional.

\subsection{Future research}
An interesting future direction of research is checking if people prefer
the schedules generated using our new system instead of the old constraint
satisfaction problem way. This could be done by feeding both systems
the same information and then asking lecturers to specify which schedule
they prefer. In order to make it even more interesting, it would also be
possible to provide an explanation for why certain lectures are scheduled
the way they are, and see if this improves peoples appreciation of the
schedules.

Another interesting area of research would be to see if the new method
could be used for general resource allocation instead of just for
scheduling. This could help in multi agent systems where multiple agents
want to use the same resource, for example a door or a hallway, but where
it is hard to find a reasonable way to give an utility for the use of the
resource. In this case, the agents could argue for why they need to use the
resource, without having to know why the other agent needs to use it, since
they will learn this during the process of argumentation. It also means
that the argumentation can happen locally, since only the agents that want
to use the resource at that time need to be involved.
