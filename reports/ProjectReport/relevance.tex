\section{Discussion}
To create the schedule we randomly pick rooms from the timetable and let agents
argue about who gets to lecture at that time slot. The random nature has several
implications for the way that the schedule is constructed. The lecture that wins
is not always a lecture that fits well into that room, it is possible that a
course with 20 students is placed in a 120 seat room because the lecturer was
the only agent that was interested in that time slot. This means that the
lecturer will teach a course to a mainly empty room, something which can be
quite demotivating even if every student shows up. Ideally a course is taught in
a room that fits best, that is, there should be as few empty seats as possible.
To enable this the slots should be filled in a more systematic way, the best
solution would be to start filling the smallest rooms with courses first. This
way small classes will always be in the smallest possible room where they fit
since the agents will try to claim whatever room that a course fits in.

Another characteristic of the random scheduling of rooms is that the system is
less likely to create the best schedule, or that it will even find a complete
schedule at all, even if it exists. A complete schedule in this case means that
all lectures that needed to be scheduled are scheduled without any conflicts.
The random nature allows for an ordering that will lead to scheduling conflicts
for single agents. If there are two agents, of which agent 1 teaches a 60 people
class, and agent 2 teaches a 120 people class, and there are two rooms
available: room 1 seats 80 people while room 2 seats 130 people. Say that we
schedule for room 2 first and agent 1 is the only grounded claim because that
agent has a higher preference for that room. It is then impossible to schedule
the class of agent 2 since the only room available seats only 80 people. So it
would've been better to schedule agent 1's class in room 1. Unfortunately
because of the random nature of our scheduling we can have this happen. A more
systematic approach could solve these kinds of conflicts before they even occur.
Scheduling the smallest rooms first would solve this. Allowing agents to make
an argument that the currently discussed room is the minimum size for their
course will also prevent this from happening, even without needing to change the
order in which the timetable is filled. The downside of this is that agents need
to know what size rooms are available, this is information that they currently
do not possess.