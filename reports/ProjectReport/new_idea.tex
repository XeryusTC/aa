\section{Method}

\subsection{Argumentation Framework}
Our argumentation framework is based on Dung's argumentation semantics
\cite{dungargfund} with special focus on the grounded semantics. In order
to make the arguments seem more natural, it is extended to also work for
support and undercuts. This method was chosen, because it was easy to
compute, but gave a powerful and easy to use interface.

Like in~\cite{dungargfund}, the basis for the framework is a graph, where
attack and support relations are given by the edges and the arguments
themselves are the nodes. The undercutting attacks go from arguments to
edges.\footnote{Because the graph library in Python that we used did not
    allow for edges that went to edges, a structural node was inserted in
    each attack and support relation} 
In the current framework it is only
possible to undercut attack and support relations, not undercutting
relations themselves. This is because of limitations of the current
implementation, not of the theoretical framework. Each relation also has a
weight, or a strength, which for supporting relationships is positive and
for attacking relationships is negative.

Because we also included support and undercutting, the grounded semantics
were extended to account for this.
\begin{defn}
    An argument is grounded iff
    \begin{enumerate}
        \item It is not attacked by a grounded argument that is not
            undercut, or
        \item The sum of the weights of the grounded, not undercut attack
            and support relations is larger than 0
    \end{enumerate}
\end{defn}

In simpler terms, this means that an argument is grounded if it has no
attacks, of if it has more support than attacks. The first clause is needed
to make sure that arguments that are not attacked are grounded. If these
arguments would not be grounded, then they do not count to determining if
the arguments they have outgoing relations to are grounded. This means that these
arguments do not have incoming attack or support relations. Therefore they
do not have more support than attacks and are not grounded either, meaning
that no argument is grounded. 

We now also need a definition for undercut, since this is important for our
definition of grounded.
\begin{defn}
    An attack or support relation is undercut iff the sum of the weights of the
    undercutting relations coming from grounded arguments is equal to or
    lower than 0.
\end{defn}


\subsection{Scheduling}

