\section{Method}

\subsection{Argumentation Framework}
Our argumentation framework is based on Dung's argumentation semantics
\cite{dungargfund} with special focus on the grounded semantics. In order
to make the arguments seem more natural, it is extended to also work for
support and undercuts. This method was chosen, because it was easy to
compute, but gave a powerful and easy to use interface.

Like in~\cite{dungargfund}, the basis for the framework is a graph, where
attack and support relations are given by the edges and the arguments
themselves are the nodes. The undercutting attacks go from arguments to
edges.\footnote{Because the graph library in Python that we used did not
    allow for edges that went to edges, a structural node was inserted in
    each attack and support relation} 
In the current framework it is only
possible to undercut attack and support relations, not undercutting
relations themselves. This is because of limitations of the current
implementation, not of the theoretical framework. Each relation also has a
weight, or a strength, which for supporting relationships is positive and
for attacking relationships is negative.

Because we also included support and undercutting, the grounded semantics
were extended to account for this.
\begin{defn}\label{def:grounded}
    An argument is grounded iff
    \begin{enumerate}
        \item It is not attacked by a grounded argument that is not
            undercut, or
        \item The sum of the weights of the grounded, not undercut attack
            and support relations is larger than 0
    \end{enumerate}
\end{defn}

In simpler terms, this means that an argument is grounded if it has no
attacks, of if it has more support than attacks. The first clause is needed
to make sure that arguments that are not attacked are grounded. If these
arguments would not be grounded, then they do not count to determining if
the arguments they have outgoing relations to are grounded. This means that these
arguments do not have incoming attack or support relations. Therefore they
do not have more support than attacks and are not grounded either, meaning
that no argument is grounded. 

We now also need a definition for undercut, since this is important for our
definition of grounded.
\begin{defn}\label{def:undercut}
    An attack or support relation is undercut iff the sum of the weights of the
    undercutting relations coming from grounded arguments is equal to or
    lower than 0.
\end{defn}

Now, to calculate whether an argument is grounded, you start with all the
arguments which have no incoming relations. These arguments are grounded by
definition. Then you go to the arguments that these arguments have
outgoing relations to. If for all the incoming relations for these
arguments it has been decided if they are grounded or not, use 
Definition~\ref{def:grounded} to determine if this argument is grounded. If
it has more grounded incoming, not undercut support relations than incoming
attack relations, then it is also grounded. Repeat with the outgoing
relations for the arguments that are now determined until done. It is also
possible to use a labelling approach where the labels are updated as
arguments and relations are added.

\subsection{Scheduling}
To create a schedule we need several pieces of information. We need to know
which rooms are available and how many students they can seat. Most courses
require a beamer while some others do not, so we also keep track of whether
there is a beamer in the room. Next we need to know the various timeslots that
each course will occupy. So a several start times and end times are given to
create blocks in which the courses will take place, these time slots are always
as shown in \autoref{tbl:timeslots}.

The final information we require is what courses the lecturers teach. For each
course we keep track of who teaches the course, how many students there are who
follow the courses, whether the course requires the use of a beamer, and how
many lectures need to be scheduled for that course. In addition to this we also
want to know the room preferences of the teachers. For this we use a number
between 0 and 1, where 0 represents that the teacher never wants to teach in
that room while 1 represents that the teacher loves the room and always wants to
teach there if he has the opportunity.

To successfully create a schedule these elements need to be combined in a way
that does not create any conflicts. In this case conflicts would be when a
teacher has to teach at two different places at the same time, multiple lectures
take place in the same room at the same time, there are more students than there
are seats in a room, or a room is unfit for the course because there is no
beamer in it while it is required.

To create the conflict free schedule we first create a $|R| \times |T| \times
5$ matrix, where $R$ is the set of rooms, $T$ is the set of available time
slots (see \autoref{tbl:timeslots}). The constant 5 comes from the number of
days we want to schedule for, in this case the days Monday through Friday. Next
we pick a random time, room and day combination from the matrix and ask each
lecturer agent if it is interested in teaching a course at this moment. The
agent decides if this is the case based on its availability, so if it is
occupied with teaching another course at that point in time then it is clearly
not able to teach at that moment as well. The agent will also check whether it
teaches a course that will fit in the room, if all courses that an agent teaches
have more students enrolled than there are seats in the room then the agent is
also not interested. After asking all agents an argument framework is created
for this room, all interested agents make a starting argument which we call a
claim. These claims start by attacking each other. Each agent determines for
itself what course it wants to start arguing for, they will always pick the
course with the most students that fits in the room that still has unscheduled
lectures.

The argument takes place over the course of several iterations. Each iteration
we ask each agent if they want to make an argument based on the current state of
the argumentation framework. Agents will support their own claim by making a
support argument. These support arguments can be one of a few types. The first
type is the size argument, in this case an agent makes the argument that the
class that they want to schedule has a certain number of students. The second
type is a beamer argument, where agents will say that they require a beamer or
not. The third type is an agent's preference for the room. After making initial
arguments agents are also allowed to start attacking and undercutting each
other's arguments. Agents will always undercut each others arguments if they
make the same argument, i.e.\ when the number of students in their course is the
same or they both require a beamer. When the agent has a better argument, like
their class being larger, than another agent it will also undercut the argument.

This process continues until no more arguments have been made. At this point we
calculate the grounded set of the argumentation framework. Often this will
result in just one claim being in the grounded set. The agent who made this
claim gets to schedule the course that they made the claim for. So this
particular room/time/day combination is now scheduled and won't be discussed
again. The winning agent keeps track of which courses have had lectures
scheduled so that they will not schedules more lectures than necessary for each
course. If there are multiple claims in the grounded set then the winner is
picked at random between these claims. When this happens it means that all
agents made equal arguments so there is no clear winner.

\begin{table}
	\centering
	\caption{Overview of the time slots that courses can be planned in.}
	\label{tbl:timeslots}
	\begin{tabular}{l|l|l}
		& Start time & End time \\ \hline
		Block 1 & 09:00 & 11:00 \\
		Block 2 & 11:00 & 13:00 \\
		Block 3 & 13:00 & 15:00 \\
		Block 4 & 15:00 & 17:00
	\end{tabular}
\end{table}

