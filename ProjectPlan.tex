\documentclass[a4paper]{article}
\usepackage{latexsym}
\usepackage[a4paper]{geometry}
\usepackage{color}
\usepackage{listings}
\usepackage[pdftex]{graphicx}
\usepackage{subfigure}

\definecolor{White}{rgb}{1,1,1}
\definecolor{Blue}{rgb}{0,0,0.5}
\definecolor{Green}{rgb}{0,0.75,0.0}
\definecolor{LightGray}{rgb}{0.6,0.6,0.6}
\definecolor{DarkGray}{rgb}{0.3,0.3,0.3}
\lstset{language=Python,
   keywords={function,uint8,uint16,uint32,double,break,case,catch,continue,else,elseif,end,for,global,if,otherwise,persistent,return,switch,try,while},
   basicstyle=\ttfamily\small,
   breaklines=true,
   keywordstyle=\bfseries\color{Blue},
   commentstyle=\itshape\color{LightGray},
   stringstyle=\color{Green},
   numbers=left,
   numberstyle=\tiny\color{DarkGray},
   stepnumber=1,
   numbersep=10pt,
   backgroundcolor=\color{White},
   tabsize=2,
   showspaces=false,
   showstringspaces=false,
   captionpos=b}

%Boldface text for type writer font
\usepackage{bold-extra} %\DeclareFontShape{OT1}{cmtt}{bx}{n}{<5><6><7><8><9><10><10.95><12><14.4><17.28><20.74><24.88>cmttb10}{}

%Break words properly at the end of a line (which isn't sloppy...)
\sloppy

%Use command \exercise for each exercise
\newcounter{exerciseCount}
\setcounter{exerciseCount}{0}
\newcommand{\exercise}[1]{\addtocounter{exerciseCount}{1} \noindent \medskip {\large \textsf{\textbf{Exercise \arabic{exerciseCount} \--- #1}}} \par}
\renewcommand{\theenumi}{\textsf{\textbf{\alph{enumi}}}}

%Use command \code for code snippets
\newcommand{\code}[1]{\textnormal{\texttt{#1}}}


\title{\textsf{Arguing Agents :} Project Plan}
\author{Pim van der Meulen \\Ren\'e Mellema \\Xeryus Stokkel}
\date{\today}

\begin{document}
\maketitle

\subsection*{Problem \& Relevance}
For this project we want to look at the problem of scheduling, or booking appropriate rooms for lectures or practicals. Due to various factors such as different room sizes, the number of course enrolments, equipment requirements and preferences of lecturers, we have a problem that contains both desires, which have the possibility of being fulfilled, and requirements, which need to be fulfilled. Together with the fact that there are resource limitations on the number of rooms and timeslots, we think that an agent based solution is useful here to create schedules that satisfy these needs.      
 
\subsection*{State of the Art}
The University of Groningen currently uses Syllabus Plus, along with a very large number of other universities and colleges throughout the world. Syllabus Plus is described as an artificial intelligence solution, which operates by first processing the foundation data, including locations, staff and preferences. Secondly, the program imports and refines all available student data. Lastly, Syllabus Plus uses a constraint satisfaction problem solver to create a schedule out of the hard constraints, soft constraints, rules and preferences that were extracted out of previous data.  

\subsection*{New Idea}
Our aim is to take a personal approach to the scheduling problem and perceive each lecturer as an agent who wants to fulfil his or hers desires. By providing various arguments an agent can defend his or hers claim on a particular room at a certain time. In turn, other agents can attack these arguments with their own, to which the first agent can respond again. 

\subsection*{Results}
We hope that by creating a forum in which agents can attack or defend various claims on a timeslot and room to create a schedule that takes all preferences and constraints into account and adequately satisfies the desires of all agents. With regards to the evaluation of our results we plan to asses our success through the number of constraints that are fulfilled and the number of desires that are satisfied.

\end{document}






